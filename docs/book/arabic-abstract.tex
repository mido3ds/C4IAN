الشبكات التلقائية المتحركة تلقي رواجاً كبيرة كمجال للبحث العلمي حيث أن لها عدداً كبيراً من التطبيقات العملية. الشبكات التلقائية المتحركة توفر طريقة للتواصل بدون بنية تحتية, و لها مستخدمون عديدون, مثل الفرق التكتيكية, و الجيوش, وفرق الطوارئ و الإنقاذ, و الروبوتات, و غيرهم.
\vspace{\baselineskip}
كيان هو نظام تواصل معتمد للفرق التكتيكية. كيان يقوم ببناء شبكة تلقائية عند الحاجة بين مراكز التحكم و الوحدات, ممكناً إياهم من التواصل بشكل سريع و آمن أثناء التحرك بحرية, بدون الحاجة إلي بنية تحتية.
\vspace{\baselineskip}
نظام كيان يمكن تشغليه علي أجهزة التواصل المحمولة الخاصة بالفرق التكتيكية, أو علي روبوتات, أو علي أي جهاز يدعم نظام تشغيل لينكس. النظام قادر علي تحديد مسار البيانات و توجيهها داخل الشبكات كبيرة الحجم المتغيرة بسرعة, بالإضافة إلي برنامجين لوحدات الفرق التكتيكية و مراكز القيادة.
\vspace{\baselineskip}
موجه البيانات في كيان يستخدم بروتوكول توجيه 
هرمي مبني على المناطق للبيانات الموجهة لوجهة واحدة, و بروتوكل توجيه متعدد عند الحاجة للبيانات الموجهة لعدة جهات, و بروتوكول نشر المعلومات لقطر معين لتوجيه البيانات الموجهة للجميع. لقد قمنا أيضاً بإضافة بعض التحسينات علي هذه البروتوكولات.
